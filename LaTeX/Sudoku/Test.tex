\documentclass[11pt, oneside]{article}   	% use "amsart" instead of "article" for AMSLaTeX format
\usepackage{geometry}                		% See geometry.pdf to learn the layout options. There are lots.
\geometry{letterpaper}                   		% ... or a4paper or a5paper or ... 
%\geometry{landscape}                		% Activate for rotated page geometry
%\usepackage[parfill]{parskip}    		% Activate to begin paragraphs with an empty line rather than an indent
\usepackage{graphicx}				% Use pdf, png, jpg, or eps§ with pdflatex; use eps in DVI mode
								% TeX will automatically convert eps --> pdf in pdflatex		
\usepackage{amssymb}

%SetFonts

%SetFonts


\title{Sudoku}
\author{Sophia Hooton}
%\date{}							% Activate to display a given date or no date

\begin{document}
\maketitle
\section{Introduction}
You are required to write a program in the language Python which manages games of "Sudoku", and also presents the user with a Graphical User Interface (GUI) to play by. Each puzzle will be hard coded and loaded from a text file. There should be three levels of difficulty: easy, medium and hard. The game follows the following principle: a square may not hold a number if that number has already been used within that square's row, column or outer square.

\section{Requirements}
\begin{description}
	\item[$\bullet$] Numbers range from 1 - 9
	\item[$\bullet$] The grid to play consists of 9 outer squares making up a 3x3 grid, with each outer square made up of a 3x3 grid of squares
\end{description}

\section{Loading the Puzzle}
Each sudoku puzzle is loaded from a file


%\subsection{}


\end{document}  